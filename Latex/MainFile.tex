
% Default to the notebook output style

    


% Inherit from the specified cell style.




    
\documentclass[11pt]{article}

    
    
    \usepackage[T1]{fontenc}
    % Nicer default font (+ math font) than Computer Modern for most use cases
    \usepackage{mathpazo}

    % Basic figure setup, for now with no caption control since it's done
    % automatically by Pandoc (which extracts ![](path) syntax from Markdown).
    \usepackage{graphicx}
    % We will generate all images so they have a width \maxwidth. This means
    % that they will get their normal width if they fit onto the page, but
    % are scaled down if they would overflow the margins.
    \makeatletter
    \def\maxwidth{\ifdim\Gin@nat@width>\linewidth\linewidth
    \else\Gin@nat@width\fi}
    \makeatother
    \let\Oldincludegraphics\includegraphics
    % Set max figure width to be 80% of text width, for now hardcoded.
    \renewcommand{\includegraphics}[1]{\Oldincludegraphics[width=.8\maxwidth]{#1}}
    % Ensure that by default, figures have no caption (until we provide a
    % proper Figure object with a Caption API and a way to capture that
    % in the conversion process - todo).
    \usepackage{caption}
    \DeclareCaptionLabelFormat{nolabel}{}
    \captionsetup{labelformat=nolabel}

    \usepackage{adjustbox} % Used to constrain images to a maximum size 
    \usepackage{xcolor} % Allow colors to be defined
    \usepackage{enumerate} % Needed for markdown enumerations to work
    \usepackage{geometry} % Used to adjust the document margins
    \usepackage{amsmath} % Equations
    \usepackage{amssymb} % Equations
    \usepackage{textcomp} % defines textquotesingle
    % Hack from http://tex.stackexchange.com/a/47451/13684:
    \AtBeginDocument{%
        \def\PYZsq{\textquotesingle}% Upright quotes in Pygmentized code
    }
    \usepackage{upquote} % Upright quotes for verbatim code
    \usepackage{eurosym} % defines \euro
    \usepackage[mathletters]{ucs} % Extended unicode (utf-8) support
    \usepackage[utf8x]{inputenc} % Allow utf-8 characters in the tex document
    \usepackage{fancyvrb} % verbatim replacement that allows latex
    \usepackage{grffile} % extends the file name processing of package graphics 
                         % to support a larger range 
    % The hyperref package gives us a pdf with properly built
    % internal navigation ('pdf bookmarks' for the table of contents,
    % internal cross-reference links, web links for URLs, etc.)
    \usepackage{hyperref}
    \usepackage{longtable} % longtable support required by pandoc >1.10
    \usepackage{booktabs}  % table support for pandoc > 1.12.2
    \usepackage[inline]{enumitem} % IRkernel/repr support (it uses the enumerate* environment)
    \usepackage[normalem]{ulem} % ulem is needed to support strikethroughs (\sout)
                                % normalem makes italics be italics, not underlines
    

    
    
    % Colors for the hyperref package
    \definecolor{urlcolor}{rgb}{0,.145,.698}
    \definecolor{linkcolor}{rgb}{.71,0.21,0.01}
    \definecolor{citecolor}{rgb}{.12,.54,.11}

    % ANSI colors
    \definecolor{ansi-black}{HTML}{3E424D}
    \definecolor{ansi-black-intense}{HTML}{282C36}
    \definecolor{ansi-red}{HTML}{E75C58}
    \definecolor{ansi-red-intense}{HTML}{B22B31}
    \definecolor{ansi-green}{HTML}{00A250}
    \definecolor{ansi-green-intense}{HTML}{007427}
    \definecolor{ansi-yellow}{HTML}{DDB62B}
    \definecolor{ansi-yellow-intense}{HTML}{B27D12}
    \definecolor{ansi-blue}{HTML}{208FFB}
    \definecolor{ansi-blue-intense}{HTML}{0065CA}
    \definecolor{ansi-magenta}{HTML}{D160C4}
    \definecolor{ansi-magenta-intense}{HTML}{A03196}
    \definecolor{ansi-cyan}{HTML}{60C6C8}
    \definecolor{ansi-cyan-intense}{HTML}{258F8F}
    \definecolor{ansi-white}{HTML}{C5C1B4}
    \definecolor{ansi-white-intense}{HTML}{A1A6B2}

    % commands and environments needed by pandoc snippets
    % extracted from the output of `pandoc -s`
    \providecommand{\tightlist}{%
      \setlength{\itemsep}{0pt}\setlength{\parskip}{0pt}}
    \DefineVerbatimEnvironment{Highlighting}{Verbatim}{commandchars=\\\{\}}
    % Add ',fontsize=\small' for more characters per line
    \newenvironment{Shaded}{}{}
    \newcommand{\KeywordTok}[1]{\textcolor[rgb]{0.00,0.44,0.13}{\textbf{{#1}}}}
    \newcommand{\DataTypeTok}[1]{\textcolor[rgb]{0.56,0.13,0.00}{{#1}}}
    \newcommand{\DecValTok}[1]{\textcolor[rgb]{0.25,0.63,0.44}{{#1}}}
    \newcommand{\BaseNTok}[1]{\textcolor[rgb]{0.25,0.63,0.44}{{#1}}}
    \newcommand{\FloatTok}[1]{\textcolor[rgb]{0.25,0.63,0.44}{{#1}}}
    \newcommand{\CharTok}[1]{\textcolor[rgb]{0.25,0.44,0.63}{{#1}}}
    \newcommand{\StringTok}[1]{\textcolor[rgb]{0.25,0.44,0.63}{{#1}}}
    \newcommand{\CommentTok}[1]{\textcolor[rgb]{0.38,0.63,0.69}{\textit{{#1}}}}
    \newcommand{\OtherTok}[1]{\textcolor[rgb]{0.00,0.44,0.13}{{#1}}}
    \newcommand{\AlertTok}[1]{\textcolor[rgb]{1.00,0.00,0.00}{\textbf{{#1}}}}
    \newcommand{\FunctionTok}[1]{\textcolor[rgb]{0.02,0.16,0.49}{{#1}}}
    \newcommand{\RegionMarkerTok}[1]{{#1}}
    \newcommand{\ErrorTok}[1]{\textcolor[rgb]{1.00,0.00,0.00}{\textbf{{#1}}}}
    \newcommand{\NormalTok}[1]{{#1}}
    
    % Additional commands for more recent versions of Pandoc
    \newcommand{\ConstantTok}[1]{\textcolor[rgb]{0.53,0.00,0.00}{{#1}}}
    \newcommand{\SpecialCharTok}[1]{\textcolor[rgb]{0.25,0.44,0.63}{{#1}}}
    \newcommand{\VerbatimStringTok}[1]{\textcolor[rgb]{0.25,0.44,0.63}{{#1}}}
    \newcommand{\SpecialStringTok}[1]{\textcolor[rgb]{0.73,0.40,0.53}{{#1}}}
    \newcommand{\ImportTok}[1]{{#1}}
    \newcommand{\DocumentationTok}[1]{\textcolor[rgb]{0.73,0.13,0.13}{\textit{{#1}}}}
    \newcommand{\AnnotationTok}[1]{\textcolor[rgb]{0.38,0.63,0.69}{\textbf{\textit{{#1}}}}}
    \newcommand{\CommentVarTok}[1]{\textcolor[rgb]{0.38,0.63,0.69}{\textbf{\textit{{#1}}}}}
    \newcommand{\VariableTok}[1]{\textcolor[rgb]{0.10,0.09,0.49}{{#1}}}
    \newcommand{\ControlFlowTok}[1]{\textcolor[rgb]{0.00,0.44,0.13}{\textbf{{#1}}}}
    \newcommand{\OperatorTok}[1]{\textcolor[rgb]{0.40,0.40,0.40}{{#1}}}
    \newcommand{\BuiltInTok}[1]{{#1}}
    \newcommand{\ExtensionTok}[1]{{#1}}
    \newcommand{\PreprocessorTok}[1]{\textcolor[rgb]{0.74,0.48,0.00}{{#1}}}
    \newcommand{\AttributeTok}[1]{\textcolor[rgb]{0.49,0.56,0.16}{{#1}}}
    \newcommand{\InformationTok}[1]{\textcolor[rgb]{0.38,0.63,0.69}{\textbf{\textit{{#1}}}}}
    \newcommand{\WarningTok}[1]{\textcolor[rgb]{0.38,0.63,0.69}{\textbf{\textit{{#1}}}}}
    
    
    % Define a nice break command that doesn't care if a line doesn't already
    % exist.
    \def\br{\hspace*{\fill} \\* }
    % Math Jax compatability definitions
    \def\gt{>}
    \def\lt{<}
    % Document parameters
    \title{ELEN3007 Probabilistic System and Signal Assignment}
    
    
    

    % Pygments definitions
    
\makeatletter
\def\PY@reset{\let\PY@it=\relax \let\PY@bf=\relax%
    \let\PY@ul=\relax \let\PY@tc=\relax%
    \let\PY@bc=\relax \let\PY@ff=\relax}
\def\PY@tok#1{\csname PY@tok@#1\endcsname}
\def\PY@toks#1+{\ifx\relax#1\empty\else%
    \PY@tok{#1}\expandafter\PY@toks\fi}
\def\PY@do#1{\PY@bc{\PY@tc{\PY@ul{%
    \PY@it{\PY@bf{\PY@ff{#1}}}}}}}
\def\PY#1#2{\PY@reset\PY@toks#1+\relax+\PY@do{#2}}

\expandafter\def\csname PY@tok@w\endcsname{\def\PY@tc##1{\textcolor[rgb]{0.73,0.73,0.73}{##1}}}
\expandafter\def\csname PY@tok@c\endcsname{\let\PY@it=\textit\def\PY@tc##1{\textcolor[rgb]{0.25,0.50,0.50}{##1}}}
\expandafter\def\csname PY@tok@cp\endcsname{\def\PY@tc##1{\textcolor[rgb]{0.74,0.48,0.00}{##1}}}
\expandafter\def\csname PY@tok@k\endcsname{\let\PY@bf=\textbf\def\PY@tc##1{\textcolor[rgb]{0.00,0.50,0.00}{##1}}}
\expandafter\def\csname PY@tok@kp\endcsname{\def\PY@tc##1{\textcolor[rgb]{0.00,0.50,0.00}{##1}}}
\expandafter\def\csname PY@tok@kt\endcsname{\def\PY@tc##1{\textcolor[rgb]{0.69,0.00,0.25}{##1}}}
\expandafter\def\csname PY@tok@o\endcsname{\def\PY@tc##1{\textcolor[rgb]{0.40,0.40,0.40}{##1}}}
\expandafter\def\csname PY@tok@ow\endcsname{\let\PY@bf=\textbf\def\PY@tc##1{\textcolor[rgb]{0.67,0.13,1.00}{##1}}}
\expandafter\def\csname PY@tok@nb\endcsname{\def\PY@tc##1{\textcolor[rgb]{0.00,0.50,0.00}{##1}}}
\expandafter\def\csname PY@tok@nf\endcsname{\def\PY@tc##1{\textcolor[rgb]{0.00,0.00,1.00}{##1}}}
\expandafter\def\csname PY@tok@nc\endcsname{\let\PY@bf=\textbf\def\PY@tc##1{\textcolor[rgb]{0.00,0.00,1.00}{##1}}}
\expandafter\def\csname PY@tok@nn\endcsname{\let\PY@bf=\textbf\def\PY@tc##1{\textcolor[rgb]{0.00,0.00,1.00}{##1}}}
\expandafter\def\csname PY@tok@ne\endcsname{\let\PY@bf=\textbf\def\PY@tc##1{\textcolor[rgb]{0.82,0.25,0.23}{##1}}}
\expandafter\def\csname PY@tok@nv\endcsname{\def\PY@tc##1{\textcolor[rgb]{0.10,0.09,0.49}{##1}}}
\expandafter\def\csname PY@tok@no\endcsname{\def\PY@tc##1{\textcolor[rgb]{0.53,0.00,0.00}{##1}}}
\expandafter\def\csname PY@tok@nl\endcsname{\def\PY@tc##1{\textcolor[rgb]{0.63,0.63,0.00}{##1}}}
\expandafter\def\csname PY@tok@ni\endcsname{\let\PY@bf=\textbf\def\PY@tc##1{\textcolor[rgb]{0.60,0.60,0.60}{##1}}}
\expandafter\def\csname PY@tok@na\endcsname{\def\PY@tc##1{\textcolor[rgb]{0.49,0.56,0.16}{##1}}}
\expandafter\def\csname PY@tok@nt\endcsname{\let\PY@bf=\textbf\def\PY@tc##1{\textcolor[rgb]{0.00,0.50,0.00}{##1}}}
\expandafter\def\csname PY@tok@nd\endcsname{\def\PY@tc##1{\textcolor[rgb]{0.67,0.13,1.00}{##1}}}
\expandafter\def\csname PY@tok@s\endcsname{\def\PY@tc##1{\textcolor[rgb]{0.73,0.13,0.13}{##1}}}
\expandafter\def\csname PY@tok@sd\endcsname{\let\PY@it=\textit\def\PY@tc##1{\textcolor[rgb]{0.73,0.13,0.13}{##1}}}
\expandafter\def\csname PY@tok@si\endcsname{\let\PY@bf=\textbf\def\PY@tc##1{\textcolor[rgb]{0.73,0.40,0.53}{##1}}}
\expandafter\def\csname PY@tok@se\endcsname{\let\PY@bf=\textbf\def\PY@tc##1{\textcolor[rgb]{0.73,0.40,0.13}{##1}}}
\expandafter\def\csname PY@tok@sr\endcsname{\def\PY@tc##1{\textcolor[rgb]{0.73,0.40,0.53}{##1}}}
\expandafter\def\csname PY@tok@ss\endcsname{\def\PY@tc##1{\textcolor[rgb]{0.10,0.09,0.49}{##1}}}
\expandafter\def\csname PY@tok@sx\endcsname{\def\PY@tc##1{\textcolor[rgb]{0.00,0.50,0.00}{##1}}}
\expandafter\def\csname PY@tok@m\endcsname{\def\PY@tc##1{\textcolor[rgb]{0.40,0.40,0.40}{##1}}}
\expandafter\def\csname PY@tok@gh\endcsname{\let\PY@bf=\textbf\def\PY@tc##1{\textcolor[rgb]{0.00,0.00,0.50}{##1}}}
\expandafter\def\csname PY@tok@gu\endcsname{\let\PY@bf=\textbf\def\PY@tc##1{\textcolor[rgb]{0.50,0.00,0.50}{##1}}}
\expandafter\def\csname PY@tok@gd\endcsname{\def\PY@tc##1{\textcolor[rgb]{0.63,0.00,0.00}{##1}}}
\expandafter\def\csname PY@tok@gi\endcsname{\def\PY@tc##1{\textcolor[rgb]{0.00,0.63,0.00}{##1}}}
\expandafter\def\csname PY@tok@gr\endcsname{\def\PY@tc##1{\textcolor[rgb]{1.00,0.00,0.00}{##1}}}
\expandafter\def\csname PY@tok@ge\endcsname{\let\PY@it=\textit}
\expandafter\def\csname PY@tok@gs\endcsname{\let\PY@bf=\textbf}
\expandafter\def\csname PY@tok@gp\endcsname{\let\PY@bf=\textbf\def\PY@tc##1{\textcolor[rgb]{0.00,0.00,0.50}{##1}}}
\expandafter\def\csname PY@tok@go\endcsname{\def\PY@tc##1{\textcolor[rgb]{0.53,0.53,0.53}{##1}}}
\expandafter\def\csname PY@tok@gt\endcsname{\def\PY@tc##1{\textcolor[rgb]{0.00,0.27,0.87}{##1}}}
\expandafter\def\csname PY@tok@err\endcsname{\def\PY@bc##1{\setlength{\fboxsep}{0pt}\fcolorbox[rgb]{1.00,0.00,0.00}{1,1,1}{\strut ##1}}}
\expandafter\def\csname PY@tok@kc\endcsname{\let\PY@bf=\textbf\def\PY@tc##1{\textcolor[rgb]{0.00,0.50,0.00}{##1}}}
\expandafter\def\csname PY@tok@kd\endcsname{\let\PY@bf=\textbf\def\PY@tc##1{\textcolor[rgb]{0.00,0.50,0.00}{##1}}}
\expandafter\def\csname PY@tok@kn\endcsname{\let\PY@bf=\textbf\def\PY@tc##1{\textcolor[rgb]{0.00,0.50,0.00}{##1}}}
\expandafter\def\csname PY@tok@kr\endcsname{\let\PY@bf=\textbf\def\PY@tc##1{\textcolor[rgb]{0.00,0.50,0.00}{##1}}}
\expandafter\def\csname PY@tok@bp\endcsname{\def\PY@tc##1{\textcolor[rgb]{0.00,0.50,0.00}{##1}}}
\expandafter\def\csname PY@tok@fm\endcsname{\def\PY@tc##1{\textcolor[rgb]{0.00,0.00,1.00}{##1}}}
\expandafter\def\csname PY@tok@vc\endcsname{\def\PY@tc##1{\textcolor[rgb]{0.10,0.09,0.49}{##1}}}
\expandafter\def\csname PY@tok@vg\endcsname{\def\PY@tc##1{\textcolor[rgb]{0.10,0.09,0.49}{##1}}}
\expandafter\def\csname PY@tok@vi\endcsname{\def\PY@tc##1{\textcolor[rgb]{0.10,0.09,0.49}{##1}}}
\expandafter\def\csname PY@tok@vm\endcsname{\def\PY@tc##1{\textcolor[rgb]{0.10,0.09,0.49}{##1}}}
\expandafter\def\csname PY@tok@sa\endcsname{\def\PY@tc##1{\textcolor[rgb]{0.73,0.13,0.13}{##1}}}
\expandafter\def\csname PY@tok@sb\endcsname{\def\PY@tc##1{\textcolor[rgb]{0.73,0.13,0.13}{##1}}}
\expandafter\def\csname PY@tok@sc\endcsname{\def\PY@tc##1{\textcolor[rgb]{0.73,0.13,0.13}{##1}}}
\expandafter\def\csname PY@tok@dl\endcsname{\def\PY@tc##1{\textcolor[rgb]{0.73,0.13,0.13}{##1}}}
\expandafter\def\csname PY@tok@s2\endcsname{\def\PY@tc##1{\textcolor[rgb]{0.73,0.13,0.13}{##1}}}
\expandafter\def\csname PY@tok@sh\endcsname{\def\PY@tc##1{\textcolor[rgb]{0.73,0.13,0.13}{##1}}}
\expandafter\def\csname PY@tok@s1\endcsname{\def\PY@tc##1{\textcolor[rgb]{0.73,0.13,0.13}{##1}}}
\expandafter\def\csname PY@tok@mb\endcsname{\def\PY@tc##1{\textcolor[rgb]{0.40,0.40,0.40}{##1}}}
\expandafter\def\csname PY@tok@mf\endcsname{\def\PY@tc##1{\textcolor[rgb]{0.40,0.40,0.40}{##1}}}
\expandafter\def\csname PY@tok@mh\endcsname{\def\PY@tc##1{\textcolor[rgb]{0.40,0.40,0.40}{##1}}}
\expandafter\def\csname PY@tok@mi\endcsname{\def\PY@tc##1{\textcolor[rgb]{0.40,0.40,0.40}{##1}}}
\expandafter\def\csname PY@tok@il\endcsname{\def\PY@tc##1{\textcolor[rgb]{0.40,0.40,0.40}{##1}}}
\expandafter\def\csname PY@tok@mo\endcsname{\def\PY@tc##1{\textcolor[rgb]{0.40,0.40,0.40}{##1}}}
\expandafter\def\csname PY@tok@ch\endcsname{\let\PY@it=\textit\def\PY@tc##1{\textcolor[rgb]{0.25,0.50,0.50}{##1}}}
\expandafter\def\csname PY@tok@cm\endcsname{\let\PY@it=\textit\def\PY@tc##1{\textcolor[rgb]{0.25,0.50,0.50}{##1}}}
\expandafter\def\csname PY@tok@cpf\endcsname{\let\PY@it=\textit\def\PY@tc##1{\textcolor[rgb]{0.25,0.50,0.50}{##1}}}
\expandafter\def\csname PY@tok@c1\endcsname{\let\PY@it=\textit\def\PY@tc##1{\textcolor[rgb]{0.25,0.50,0.50}{##1}}}
\expandafter\def\csname PY@tok@cs\endcsname{\let\PY@it=\textit\def\PY@tc##1{\textcolor[rgb]{0.25,0.50,0.50}{##1}}}

\def\PYZbs{\char`\\}
\def\PYZus{\char`\_}
\def\PYZob{\char`\{}
\def\PYZcb{\char`\}}
\def\PYZca{\char`\^}
\def\PYZam{\char`\&}
\def\PYZlt{\char`\<}
\def\PYZgt{\char`\>}
\def\PYZsh{\char`\#}
\def\PYZpc{\char`\%}
\def\PYZdl{\char`\$}
\def\PYZhy{\char`\-}
\def\PYZsq{\char`\'}
\def\PYZdq{\char`\"}
\def\PYZti{\char`\~}
% for compatibility with earlier versions
\def\PYZat{@}
\def\PYZlb{[}
\def\PYZrb{]}
\makeatother


    % Exact colors from NB
    \definecolor{incolor}{rgb}{0.0, 0.0, 0.5}
    \definecolor{outcolor}{rgb}{0.545, 0.0, 0.0}



    
    % Prevent overflowing lines due to hard-to-break entities
    \sloppy 
    % Setup hyperref package
    \hypersetup{
      breaklinks=true,  % so long urls are correctly broken across lines
      colorlinks=true,
      urlcolor=urlcolor,
      linkcolor=linkcolor,
      citecolor=citecolor,
      }
    % Slightly bigger margins than the latex defaults
    
    \geometry{verbose,tmargin=1in,bmargin=1in,lmargin=1in,rmargin=1in}
    
    

    \begin{document}
    
    
    \maketitle
    
    

    
    \section{ELEN3007
ICAO-Codes-Weather-analysis}\label{elen3007-icao-codes-weather-analysis}

\subparagraph{This Script imports the weather data for the Chulman
Airport (UELL) weather station, in Russia and then answers question
outlined.}\label{this-script-imports-the-weather-data-for-the-chulman-airport-uell-weather-station-in-russia-and-then-answers-question-outlined.}

Each question that has a discussion of results is analised in the
respective question markdown block before the code block.

The Git repo, storing this code, can be seen here:
https://github.com/SoIidarity/ICAO-Weather-Analysis-Python

This main file, showing all code execution, can be seen here:
https://github.com/SoIidarity/ICAO-Weather-Analysis-Python/blob/master/MainFile.ipynb
\\

Both a PDF version and HTML version of this report are included in this
submission. For an optimal viewing experience, please view the HTML
version as the formatting is persistent from the Jupyter notebook.
\\

Please note that this file was generated using the built in Jupyter notebook Latex export function. As a result, some graphics are not included and so to view all content presented in this report, please view the HTML version also included in the submission.
\\

    Import libraries and such needed for program execution.

    \begin{Verbatim}[commandchars=\\\{\}]
{\color{incolor}In [{\color{incolor}1}]:} \PY{k+kn}{import} \PY{n+nn}{numpy} \PY{k}{as} \PY{n+nn}{np}
        \PY{k+kn}{import} \PY{n+nn}{seaborn} \PY{k}{as} \PY{n+nn}{sns}
        
        \PY{k+kn}{import} \PY{n+nn}{pandas} \PY{k}{as} \PY{n+nn}{pd}
        \PY{k+kn}{from} \PY{n+nn}{statsmodels}\PY{n+nn}{.}\PY{n+nn}{graphics}\PY{n+nn}{.}\PY{n+nn}{tsaplots} \PY{k}{import} \PY{n}{plot\PYZus{}acf}
        
        \PY{k+kn}{import} \PY{n+nn}{scipy}\PY{n+nn}{.}\PY{n+nn}{stats}
        \PY{k+kn}{import} \PY{n+nn}{matplotlib}\PY{n+nn}{.}\PY{n+nn}{pyplot} \PY{k}{as} \PY{n+nn}{plt}
        \PY{k+kn}{from} \PY{n+nn}{IPython}\PY{n+nn}{.}\PY{n+nn}{display} \PY{k}{import} \PY{n}{HTML}\PY{p}{,} \PY{n}{display}
        
        \PY{k+kn}{from} \PY{n+nn}{io} \PY{k}{import} \PY{n}{BytesIO}
        \PY{k+kn}{from} \PY{n+nn}{base64} \PY{k}{import} \PY{n}{b64encode}
        \PY{k+kn}{import} \PY{n+nn}{scipy}\PY{n+nn}{.}\PY{n+nn}{misc} \PY{k}{as} \PY{n+nn}{smp}
        
        \PY{k+kn}{from} \PY{n+nn}{mpl\PYZus{}toolkits}\PY{n+nn}{.}\PY{n+nn}{mplot3d} \PY{k}{import} \PY{n}{Axes3D}
        \PY{k+kn}{from} \PY{n+nn}{matplotlib} \PY{k}{import} \PY{n}{cm}
        \PY{k+kn}{from} \PY{n+nn}{matplotlib}\PY{n+nn}{.}\PY{n+nn}{ticker} \PY{k}{import} \PY{n}{LinearLocator}
        \PY{k+kn}{from} \PY{n+nn}{scipy}\PY{n+nn}{.}\PY{n+nn}{stats} \PY{k}{import} \PY{n}{norm}
        
        \PY{n}{plt}\PY{o}{.}\PY{n}{rcParams}\PY{p}{[}\PY{l+s+s1}{\PYZsq{}}\PY{l+s+s1}{figure.figsize}\PY{l+s+s1}{\PYZsq{}}\PY{p}{]} \PY{o}{=} \PY{p}{(}\PY{l+m+mi}{24}\PY{p}{,} \PY{l+m+mi}{6}\PY{p}{)}
\end{Verbatim}


    First, define some useful functions. These are used in the printing of
data later on.

    \begin{Verbatim}[commandchars=\\\{\}]
{\color{incolor}In [{\color{incolor}2}]:} \PY{k}{def} \PY{n+nf}{printMatrix}\PY{p}{(}\PY{n}{data}\PY{p}{)}\PY{p}{:}  \PY{c+c1}{\PYZsh{}used to print matricies to HTML}
            \PY{n}{display}\PY{p}{(}\PY{n}{HTML}\PY{p}{(}
                \PY{l+s+s1}{\PYZsq{}}\PY{l+s+s1}{\PYZlt{}table\PYZgt{}\PYZlt{}tr\PYZgt{}}\PY{l+s+si}{\PYZob{}\PYZcb{}}\PY{l+s+s1}{\PYZlt{}/tr\PYZgt{}\PYZlt{}/table\PYZgt{}}\PY{l+s+s1}{\PYZsq{}}\PY{o}{.}\PY{n}{format}\PY{p}{(}
                    \PY{l+s+s1}{\PYZsq{}}\PY{l+s+s1}{\PYZlt{}/tr\PYZgt{}\PYZlt{}tr\PYZgt{}}\PY{l+s+s1}{\PYZsq{}}\PY{o}{.}\PY{n}{join}\PY{p}{(}
                        \PY{l+s+s1}{\PYZsq{}}\PY{l+s+s1}{\PYZlt{}td\PYZgt{}}\PY{l+s+si}{\PYZob{}\PYZcb{}}\PY{l+s+s1}{\PYZlt{}/td\PYZgt{}}\PY{l+s+s1}{\PYZsq{}}
                        \PY{o}{.}\PY{n}{format}\PY{p}{(}\PY{l+s+s1}{\PYZsq{}}\PY{l+s+s1}{\PYZlt{}/td\PYZgt{}\PYZlt{}td\PYZgt{}}\PY{l+s+s1}{\PYZsq{}}
                        \PY{o}{.}\PY{n}{join}\PY{p}{(}\PY{n+nb}{str}\PY{p}{(}\PY{n}{\PYZus{}}\PY{p}{)} \PY{k}{for} \PY{n}{\PYZus{}} \PY{o+ow}{in} \PY{n}{row}\PY{p}{)}\PY{p}{)} \PY{k}{for} \PY{n}{row} \PY{o+ow}{in} \PY{n}{data}\PY{p}{)}
                \PY{p}{)}
            \PY{p}{)}\PY{p}{)}
        
        \PY{k}{def} \PY{n+nf}{printText}\PY{p}{(}\PY{n}{text}\PY{p}{)}\PY{p}{:}
            \PY{n}{display}\PY{p}{(}\PY{n}{HTML}\PY{p}{(}\PY{l+s+s1}{\PYZsq{}}\PY{l+s+s1}{\PYZlt{}p\PYZgt{}}\PY{l+s+s1}{\PYZsq{}} \PY{o}{+} \PY{n}{text} \PY{o}{+} \PY{l+s+s1}{\PYZsq{}}\PY{l+s+s1}{\PYZlt{}p\PYZgt{}}\PY{l+s+s1}{\PYZsq{}}\PY{p}{)}\PY{p}{)}
        
        \PY{k}{def} \PY{n+nf}{displayHTML}\PY{p}{(}\PY{n}{html}\PY{p}{)}\PY{p}{:}
            \PY{n}{display}\PY{p}{(}\PY{n}{HTML}\PY{p}{(}\PY{n}{html}\PY{p}{)}\PY{p}{)}
            
        \PY{k}{def} \PY{n+nf}{drawImg}\PY{p}{(}\PY{n}{img}\PY{p}{)}\PY{p}{:}
            \PY{n}{b} \PY{o}{=} \PY{n}{BytesIO}\PY{p}{(}\PY{p}{)}
            \PY{n}{img}\PY{o}{.}\PY{n}{save}\PY{p}{(}\PY{n}{b}\PY{p}{,} \PY{n+nb}{format}\PY{o}{=}\PY{l+s+s1}{\PYZsq{}}\PY{l+s+s1}{png}\PY{l+s+s1}{\PYZsq{}}\PY{p}{)}
            \PY{n}{displayHTML}\PY{p}{(}\PY{l+s+s2}{\PYZdq{}}\PY{l+s+s2}{\PYZlt{}img src=}\PY{l+s+s2}{\PYZsq{}}\PY{l+s+s2}{data:image/png;base64,}\PY{l+s+si}{\PYZob{}0\PYZcb{}}\PY{l+s+s2}{\PYZsq{}}\PY{l+s+s2}{/\PYZgt{}}\PY{l+s+s2}{\PYZdq{}}
                        \PY{o}{.}\PY{n}{format}\PY{p}{(}\PY{n}{b64encode}\PY{p}{(}\PY{n}{b}\PY{o}{.}\PY{n}{getvalue}\PY{p}{(}\PY{p}{)}\PY{p}{)}\PY{o}{.}\PY{n}{decode}\PY{p}{(}\PY{l+s+s1}{\PYZsq{}}\PY{l+s+s1}{utf\PYZhy{}8}\PY{l+s+s1}{\PYZsq{}}\PY{p}{)}\PY{p}{)}\PY{p}{)}
\end{Verbatim}

    \subsection{Question 1}\label{question-1}

Import data from text files. These are stored as csvs in the "Data"
directory in the repo. This CSV is formatted as: \{Unit Timestamp, max
Temp, avg Temp, min Temp\}. Each CSV is read into a matrix. These
matricies are then added to a vector so they can be itterated through
later on.

    \begin{Verbatim}[commandchars=\\\{\}]
{\color{incolor}In [{\color{incolor}3}]:} \PY{n}{w1995} \PY{o}{=} \PY{n}{np}\PY{o}{.}\PY{n}{genfromtxt}\PY{p}{(}\PY{l+s+s1}{\PYZsq{}}\PY{l+s+s1}{Data/1995.csv}\PY{l+s+s1}{\PYZsq{}}\PY{p}{,} \PY{n}{delimiter}\PY{o}{=}\PY{l+s+s1}{\PYZsq{}}\PY{l+s+s1}{,}\PY{l+s+s1}{\PYZsq{}}\PY{p}{)}
        \PY{n}{w2000} \PY{o}{=} \PY{n}{np}\PY{o}{.}\PY{n}{genfromtxt}\PY{p}{(}\PY{l+s+s1}{\PYZsq{}}\PY{l+s+s1}{Data/2000.csv}\PY{l+s+s1}{\PYZsq{}}\PY{p}{,} \PY{n}{delimiter}\PY{o}{=}\PY{l+s+s1}{\PYZsq{}}\PY{l+s+s1}{,}\PY{l+s+s1}{\PYZsq{}}\PY{p}{)}
        \PY{n}{w2005} \PY{o}{=} \PY{n}{np}\PY{o}{.}\PY{n}{genfromtxt}\PY{p}{(}\PY{l+s+s1}{\PYZsq{}}\PY{l+s+s1}{Data/2005.csv}\PY{l+s+s1}{\PYZsq{}}\PY{p}{,} \PY{n}{delimiter}\PY{o}{=}\PY{l+s+s1}{\PYZsq{}}\PY{l+s+s1}{,}\PY{l+s+s1}{\PYZsq{}}\PY{p}{)}
        \PY{n}{w2010} \PY{o}{=} \PY{n}{np}\PY{o}{.}\PY{n}{genfromtxt}\PY{p}{(}\PY{l+s+s1}{\PYZsq{}}\PY{l+s+s1}{Data/2010.csv}\PY{l+s+s1}{\PYZsq{}}\PY{p}{,} \PY{n}{delimiter}\PY{o}{=}\PY{l+s+s1}{\PYZsq{}}\PY{l+s+s1}{,}\PY{l+s+s1}{\PYZsq{}}\PY{p}{)}
        \PY{n}{w2015} \PY{o}{=} \PY{n}{np}\PY{o}{.}\PY{n}{genfromtxt}\PY{p}{(}\PY{l+s+s1}{\PYZsq{}}\PY{l+s+s1}{Data/2015.csv}\PY{l+s+s1}{\PYZsq{}}\PY{p}{,} \PY{n}{delimiter}\PY{o}{=}\PY{l+s+s1}{\PYZsq{}}\PY{l+s+s1}{,}\PY{l+s+s1}{\PYZsq{}}\PY{p}{)}
        
        \PY{n}{weatherData} \PY{o}{=} \PY{p}{[}\PY{n}{w1995}\PY{p}{,} \PY{n}{w2000}\PY{p}{,} \PY{n}{w2005}\PY{p}{,} \PY{n}{w2010}\PY{p}{,} \PY{n}{w2015}\PY{p}{]}
\end{Verbatim}

    \subsection{Question 2}\label{question-2}

Next, identify the minimum, maximum, mean and standard deviation for
each year. As the matricies are in a vector, we can do this sequentially
in a loop. These outputs are produced in a matrix, where the columns are
the years (1995,2000,2005,2010,2015) and the rows are the minimum,
maximum, mean and standard deviation for each year. This can be seen in
the table below.

\subsubsection{Comment on findings}\label{comment-on-findings}

The minimum, maximum, mean and standard deviations are very similar for
each year. this is to be expected as they were taken from the same
location. However, as time went on over the past 20 years, it seems that
the averages became walmer with both the maximums and minimums
increasing over time. There are outlyers in this trend as seen in 2010's
minimum tempreture. This trend is also verified by looking at the mean
where each year is shown to be getting walmer.

    \begin{Verbatim}[commandchars=\\\{\}]
{\color{incolor}In [{\color{incolor}4}]:} \PY{n}{dataValues} \PY{o}{=} \PY{n}{np}\PY{o}{.}\PY{n}{zeros}\PY{p}{(}\PY{p}{(}\PY{l+m+mi}{4}\PY{p}{,} \PY{l+m+mi}{5}\PY{p}{)}\PY{p}{)}
        \PY{n}{counter} \PY{o}{=} \PY{l+m+mi}{0}\PY{p}{;}
        \PY{k}{for} \PY{n}{year} \PY{o+ow}{in} \PY{n}{weatherData}\PY{p}{:}
            \PY{n}{dataValues}\PY{p}{[}\PY{l+m+mi}{0}\PY{p}{,} \PY{n}{counter}\PY{p}{]} \PY{o}{=} \PY{n}{year}\PY{p}{[}\PY{p}{:}\PY{p}{,} \PY{l+m+mi}{3}\PY{p}{]}\PY{o}{.}\PY{n}{min}\PY{p}{(}\PY{p}{)}  \PY{c+c1}{\PYZsh{}max of max values}
            \PY{n}{dataValues}\PY{p}{[}\PY{l+m+mi}{1}\PY{p}{,} \PY{n}{counter}\PY{p}{]} \PY{o}{=} \PY{n}{year}\PY{p}{[}\PY{p}{:}\PY{p}{,} \PY{l+m+mi}{1}\PY{p}{]}\PY{o}{.}\PY{n}{max}\PY{p}{(}\PY{p}{)}  \PY{c+c1}{\PYZsh{}min of min values}
            \PY{n}{dataValues}\PY{p}{[}\PY{l+m+mi}{2}\PY{p}{,} \PY{n}{counter}\PY{p}{]} \PY{o}{=} \PY{n+nb}{round}\PY{p}{(}\PY{n}{year}\PY{p}{[}\PY{p}{:}\PY{p}{,} \PY{l+m+mi}{2}\PY{p}{]}\PY{o}{.}\PY{n}{mean}\PY{p}{(}\PY{p}{)}\PY{p}{,} \PY{l+m+mi}{2}\PY{p}{)}  \PY{c+c1}{\PYZsh{}average of average values}
            \PY{n}{dataValues}\PY{p}{[}\PY{l+m+mi}{3}\PY{p}{,} \PY{n}{counter}\PY{p}{]} \PY{o}{=} \PY{n+nb}{round}\PY{p}{(}\PY{n}{year}\PY{p}{[}\PY{p}{:}\PY{p}{,} \PY{l+m+mi}{2}\PY{p}{]}\PY{o}{.}\PY{n}{std}\PY{p}{(}\PY{p}{)}\PY{p}{,} \PY{l+m+mi}{2}\PY{p}{)}  \PY{c+c1}{\PYZsh{}Standard deviation of average values}
            \PY{n}{counter} \PY{o}{=} \PY{n}{counter} \PY{o}{+} \PY{l+m+mi}{1}\PY{p}{;}
        \PY{c+c1}{\PYZsh{} printMatrix(dataValues)}
        
        
        \PY{n}{displayHTML}\PY{p}{(}\PY{n}{pd}\PY{o}{.}\PY{n}{DataFrame}\PY{p}{(}\PY{n}{dataValues}\PY{p}{,}
                                 \PY{n}{columns}\PY{o}{=}\PY{p}{[}\PY{l+s+s1}{\PYZsq{}}\PY{l+s+s1}{1995}\PY{l+s+s1}{\PYZsq{}}\PY{p}{,} \PY{l+s+s1}{\PYZsq{}}\PY{l+s+s1}{2000}\PY{l+s+s1}{\PYZsq{}}\PY{p}{,} \PY{l+s+s1}{\PYZsq{}}\PY{l+s+s1}{2005}\PY{l+s+s1}{\PYZsq{}}\PY{p}{,} \PY{l+s+s1}{\PYZsq{}}\PY{l+s+s1}{2010}\PY{l+s+s1}{\PYZsq{}}\PY{p}{,} \PY{l+s+s1}{\PYZsq{}}\PY{l+s+s1}{2015}\PY{l+s+s1}{\PYZsq{}}\PY{p}{]}\PY{p}{,}
                                 \PY{n}{index}\PY{o}{=}\PY{p}{[}\PY{l+s+s1}{\PYZsq{}}\PY{l+s+s1}{Minimum}\PY{l+s+s1}{\PYZsq{}}\PY{p}{,} \PY{l+s+s1}{\PYZsq{}}\PY{l+s+s1}{Maximum}\PY{l+s+s1}{\PYZsq{}}\PY{p}{,} \PY{l+s+s1}{\PYZsq{}}\PY{l+s+s1}{Mean}\PY{l+s+s1}{\PYZsq{}}\PY{p}{,} \PY{l+s+s1}{\PYZsq{}}\PY{l+s+s1}{Standard Deviation}\PY{l+s+s1}{\PYZsq{}}\PY{p}{]}
                                 \PY{p}{)}\PY{o}{.}\PY{n}{to\PYZus{}html}\PY{p}{(}\PY{p}{)}\PY{p}{)}
\end{Verbatim}

    
    \begin{verbatim}
<IPython.core.display.HTML object>
    \end{verbatim}

    
    \subsection{Question 3}\label{question-3}

Each Years probability distribution functions are now plotted, for each
year, on the same set of axes. To achive this, a fixed plot of
univariate distributions is generated. This is done with the Seaborn
displot function.

\subsubsection{Comment on findings}\label{comment-on-findings}

Each year has a similar general probobility distribution function. They
apear to be close to normal in shape. general average trends can also be
seen through this graph such as 2005 had higher distribution of both
highs and lows, with a lower distribution in the middel. This indicates
more extreme weather during this year.

    \begin{Verbatim}[commandchars=\\\{\}]
{\color{incolor}In [{\color{incolor}5}]:} \PY{n}{sns}\PY{o}{.}\PY{n}{set\PYZus{}style}\PY{p}{(}\PY{l+s+s1}{\PYZsq{}}\PY{l+s+s1}{whitegrid}\PY{l+s+s1}{\PYZsq{}}\PY{p}{)}
        \PY{n}{counter} \PY{o}{=} \PY{l+m+mi}{0}\PY{p}{;}
        \PY{k}{for} \PY{n}{year} \PY{o+ow}{in} \PY{n}{weatherData}\PY{p}{:}
            \PY{n}{sns}\PY{o}{.}\PY{n}{distplot}\PY{p}{(}\PY{n}{year}\PY{p}{[}\PY{p}{:}\PY{p}{,} \PY{l+m+mi}{2}\PY{p}{]}\PY{p}{,} \PY{n}{hist}\PY{o}{=}\PY{k+kc}{False}\PY{p}{,}
                         \PY{n}{label}\PY{o}{=}\PY{l+m+mi}{1995} \PY{o}{+} \PY{n}{counter} \PY{o}{*} \PY{l+m+mi}{5}\PY{p}{,} 
                         \PY{n}{axlabel}\PY{o}{=}\PY{l+s+s2}{\PYZdq{}}\PY{l+s+s2}{Temperature (C)}\PY{l+s+s2}{\PYZdq{}}\PY{p}{)}
            \PY{n}{counter} \PY{o}{=} \PY{n}{counter} \PY{o}{+} \PY{l+m+mi}{1}\PY{p}{;}
        \PY{n}{plt}\PY{o}{.}\PY{n}{title}\PY{p}{(}\PY{l+s+s1}{\PYZsq{}}\PY{l+s+s1}{Probability Distribution Function}\PY{l+s+s1}{\PYZsq{}}\PY{p}{)}
        \PY{n}{plt}\PY{o}{.}\PY{n}{show}\PY{p}{(}\PY{p}{)}
\end{Verbatim}

    \begin{center}
    \adjustimage{max size={0.9\linewidth}{0.9\paperheight}}{output_10_0.png}
    \end{center}
    { \hspace*{\fill} \\}
    
    \subsection{Question 4}\label{question-4}

The cross-correlation between each year's annual temperatures is now
calculated. This is shown as a matrix output Note here that we are
slicing the data on each year to ignore the leap year in 2000. This is
done as the correlation needs matricies of the same dimension.

Additionally note that we need to normalise the data. This is the same
as using the numpy corrfoef function.

\subsubsection{Comment on findings}\label{comment-on-findings}

Through the cross-correlation of the presented data, one can see the
relationship between each years weather. The closer these values are to
1, the more closely correlated the data is. the highest correlation is
seen between 2010 and 1995 intrestingly enough with a 94\% correlation.
The main diagonal of 1's indicates the full correlation, with each year
corelated against its self.

    \begin{Verbatim}[commandchars=\\\{\}]
{\color{incolor}In [{\color{incolor}6}]:} \PY{n}{autoCorrelation} \PY{o}{=} \PY{n}{np}\PY{o}{.}\PY{n}{zeros}\PY{p}{(}\PY{p}{(}\PY{l+m+mi}{5}\PY{p}{,} \PY{l+m+mi}{5}\PY{p}{)}\PY{p}{)}
        \PY{k}{for} \PY{n}{x} \PY{o+ow}{in} \PY{n+nb}{range}\PY{p}{(}\PY{l+m+mi}{0}\PY{p}{,} \PY{l+m+mi}{5}\PY{p}{)}\PY{p}{:}
            \PY{k}{for} \PY{n}{y} \PY{o+ow}{in} \PY{n+nb}{range}\PY{p}{(}\PY{l+m+mi}{0}\PY{p}{,} \PY{l+m+mi}{5}\PY{p}{)}\PY{p}{:}
                \PY{n}{a} \PY{o}{=} \PY{n}{weatherData}\PY{p}{[}\PY{n}{x}\PY{p}{]}\PY{p}{[}\PY{l+m+mi}{0}\PY{p}{:}\PY{l+m+mi}{365}\PY{p}{,} \PY{l+m+mi}{2}\PY{p}{]}
                \PY{n}{b} \PY{o}{=} \PY{n}{weatherData}\PY{p}{[}\PY{n}{y}\PY{p}{]}\PY{p}{[}\PY{l+m+mi}{0}\PY{p}{:}\PY{l+m+mi}{365}\PY{p}{,} \PY{l+m+mi}{2}\PY{p}{]}
                \PY{n}{a} \PY{o}{=} \PY{p}{(}\PY{n}{a} \PY{o}{\PYZhy{}} \PY{n}{np}\PY{o}{.}\PY{n}{mean}\PY{p}{(}\PY{n}{a}\PY{p}{)}\PY{p}{)} \PY{o}{/} \PY{p}{(}\PY{n}{np}\PY{o}{.}\PY{n}{std}\PY{p}{(}\PY{n}{a}\PY{p}{)} \PY{o}{*} \PY{n+nb}{len}\PY{p}{(}\PY{n}{a}\PY{p}{)}\PY{p}{)}
                \PY{n}{b} \PY{o}{=} \PY{p}{(}\PY{n}{b} \PY{o}{\PYZhy{}} \PY{n}{np}\PY{o}{.}\PY{n}{mean}\PY{p}{(}\PY{n}{b}\PY{p}{)}\PY{p}{)} \PY{o}{/} \PY{p}{(}\PY{n}{np}\PY{o}{.}\PY{n}{std}\PY{p}{(}\PY{n}{b}\PY{p}{)}\PY{p}{)}
                \PY{n}{autoCorrelation}\PY{p}{[}\PY{n}{x}\PY{p}{,} \PY{n}{y}\PY{p}{]} \PY{o}{=} \PY{n}{np}\PY{o}{.}\PY{n}{correlate}\PY{p}{(}\PY{n}{a}\PY{p}{,} \PY{n}{b}\PY{p}{)}
                \PY{n}{columns}\PY{o}{=}\PY{p}{[}\PY{l+s+s1}{\PYZsq{}}\PY{l+s+s1}{1995}\PY{l+s+s1}{\PYZsq{}}\PY{p}{,} \PY{l+s+s1}{\PYZsq{}}\PY{l+s+s1}{2000}\PY{l+s+s1}{\PYZsq{}}\PY{p}{,} \PY{l+s+s1}{\PYZsq{}}\PY{l+s+s1}{2005}\PY{l+s+s1}{\PYZsq{}}\PY{p}{,} \PY{l+s+s1}{\PYZsq{}}\PY{l+s+s1}{2010}\PY{l+s+s1}{\PYZsq{}}\PY{p}{,} \PY{l+s+s1}{\PYZsq{}}\PY{l+s+s1}{2015}\PY{l+s+s1}{\PYZsq{}}\PY{p}{]}
                \PY{n}{index}\PY{o}{=}\PY{p}{[}\PY{l+s+s1}{\PYZsq{}}\PY{l+s+s1}{1995}\PY{l+s+s1}{\PYZsq{}}\PY{p}{,} \PY{l+s+s1}{\PYZsq{}}\PY{l+s+s1}{2000}\PY{l+s+s1}{\PYZsq{}}\PY{p}{,} \PY{l+s+s1}{\PYZsq{}}\PY{l+s+s1}{2005}\PY{l+s+s1}{\PYZsq{}}\PY{p}{,} \PY{l+s+s1}{\PYZsq{}}\PY{l+s+s1}{2010}\PY{l+s+s1}{\PYZsq{}}\PY{p}{,} \PY{l+s+s1}{\PYZsq{}}\PY{l+s+s1}{2015}\PY{l+s+s1}{\PYZsq{}}\PY{p}{]}
        \PY{n}{displayHTML}\PY{p}{(}\PY{n}{pd}\PY{o}{.}\PY{n}{DataFrame}\PY{p}{(}\PY{n}{autoCorrelation}\PY{p}{,}
                                 \PY{n}{columns}\PY{p}{,}
                                 \PY{n}{index}\PY{p}{)}\PY{o}{.}\PY{n}{to\PYZus{}html}\PY{p}{(}\PY{p}{)}\PY{p}{)}
\end{Verbatim}

    
    \begin{verbatim}
<IPython.core.display.HTML object>
    \end{verbatim}

    
    \subsection{Question 5}\label{question-5}

The autocorrelation function for each year's data, where τ ranges from 0
to 364 is now generated.

Confidence intervals are drawn as a cone. By default, this is set to a
95\% confidence interval, suggesting that correlation values outside of
this code are very likely a correlation and not a statistical fluke.

First, each Autocorrelation is drawn on its own graph. After, They are
overlayed.

\subsubsection{Comment on findings}\label{comment-on-findings}

This autocorrelation shows how each year is related to a time-shifted
version of itself. This can be seen as an effective convolution process.
From this, one can see how each part of the year is related to the rest
of the year for a given shift amount τ. For example, at τ=0, the
magnitude is 1. At this time, no shifting has occurred. At approximately
τ=150, there is the highest negative correlation seen. This is due to
the correlation between the middle of summer and winter of the two data
sets.

    \begin{Verbatim}[commandchars=\\\{\}]
{\color{incolor}In [{\color{incolor}12}]:} \PY{n}{counter} \PY{o}{=} \PY{l+m+mi}{0}\PY{p}{;}
         \PY{k}{for} \PY{n}{year} \PY{o+ow}{in} \PY{n}{weatherData}\PY{p}{:}
         
             \PY{k}{if} \PY{p}{(}\PY{n}{counter} \PY{o}{==} \PY{l+m+mi}{0}\PY{p}{)}\PY{p}{:}  \PY{c+c1}{\PYZsh{}on the first loop, generate the subplots}
                 \PY{n}{fig}\PY{p}{,} \PY{n}{axs} \PY{o}{=} \PY{n}{plt}\PY{o}{.}\PY{n}{subplots}\PY{p}{(}\PY{l+m+mi}{1}\PY{p}{,} \PY{l+m+mi}{2}\PY{p}{)}
             \PY{k}{if} \PY{n}{counter} \PY{o}{==} \PY{l+m+mi}{4}\PY{p}{:}  \PY{c+c1}{\PYZsh{} if we are at the last figure, we want it to be on its own line}
                 \PY{n}{plot\PYZus{}acf}\PY{p}{(}\PY{n}{year}\PY{p}{[}\PY{p}{:}\PY{p}{,} \PY{l+m+mi}{2}\PY{p}{]}\PY{p}{,} 
                          \PY{n}{title}\PY{o}{=}\PY{l+s+s2}{\PYZdq{}}\PY{l+s+s2}{Autocorrelation for }\PY{l+s+si}{\PYZob{}\PYZcb{}}\PY{l+s+s2}{\PYZdq{}}
                          \PY{o}{.}\PY{n}{format}\PY{p}{(}\PY{l+m+mi}{1995} \PY{o}{+} \PY{l+m+mi}{5} \PY{o}{*} \PY{n}{counter}\PY{p}{)}\PY{p}{)}
                 \PY{n}{plt}\PY{o}{.}\PY{n}{show}\PY{p}{(}\PY{p}{)}
             \PY{k}{else}\PY{p}{:}
                 \PY{n}{plot\PYZus{}acf}\PY{p}{(}\PY{n}{year}\PY{p}{[}\PY{p}{:}\PY{p}{,} \PY{l+m+mi}{2}\PY{p}{]}\PY{p}{,} 
                          \PY{n}{title}\PY{o}{=}\PY{l+s+s2}{\PYZdq{}}\PY{l+s+s2}{Autocorrelation for }\PY{l+s+si}{\PYZob{}\PYZcb{}}\PY{l+s+s2}{\PYZdq{}}
                          \PY{o}{.}\PY{n}{format}\PY{p}{(}\PY{l+m+mi}{1995} \PY{o}{+} \PY{l+m+mi}{5} \PY{o}{*} \PY{n}{counter}\PY{p}{)}\PY{p}{,} 
                          \PY{n}{ax}\PY{o}{=}\PY{n}{axs}\PY{p}{[}\PY{n}{counter} \PY{o}{\PYZpc{}} \PY{l+m+mi}{2}\PY{p}{]}\PY{p}{)}
             \PY{k}{if} \PY{n}{counter} \PY{o}{\PYZpc{}} \PY{l+m+mi}{2} \PY{o}{==} \PY{l+m+mi}{1}\PY{p}{:}  \PY{c+c1}{\PYZsh{} every two figures, we need to generate a new row}
                 \PY{n}{plt}\PY{o}{.}\PY{n}{show}\PY{p}{(}\PY{p}{)}
                 \PY{k}{if} \PY{p}{(}\PY{n}{counter} \PY{o}{\PYZlt{}} \PY{l+m+mi}{2}\PY{p}{)}\PY{p}{:}  \PY{c+c1}{\PYZsh{}a new sub plot is needed on second row}
                     \PY{n}{fig}\PY{p}{,} \PY{n}{axs} \PY{o}{=} \PY{n}{plt}\PY{o}{.}\PY{n}{subplots}\PY{p}{(}\PY{l+m+mi}{1}\PY{p}{,} \PY{l+m+mi}{2}\PY{p}{)}
             \PY{n}{counter} \PY{o}{=} \PY{n}{counter} \PY{o}{+} \PY{l+m+mi}{1}    
         
         \PY{n}{counter}\PY{o}{=}\PY{l+m+mi}{0}\PY{p}{;}
         \PY{n}{fig}\PY{p}{,} \PY{n}{axs} \PY{o}{=} \PY{n}{plt}\PY{o}{.}\PY{n}{subplots}\PY{p}{(}\PY{l+m+mi}{1}\PY{p}{,} \PY{l+m+mi}{1}\PY{p}{)}
         \PY{k}{for} \PY{n}{year} \PY{o+ow}{in} \PY{n}{weatherData}\PY{p}{:}
             
             \PY{n}{plot\PYZus{}acf}\PY{p}{(}\PY{n}{year}\PY{p}{[}\PY{p}{:}\PY{p}{,} \PY{l+m+mi}{2}\PY{p}{]}\PY{p}{,}
                      \PY{n}{title}\PY{o}{=}\PY{l+s+s2}{\PYZdq{}}\PY{l+s+s2}{Autocorrelation All 5 years together}\PY{l+s+s2}{\PYZdq{}}
                      \PY{o}{.}\PY{n}{format}\PY{p}{(}\PY{l+m+mi}{1995} \PY{o}{+} \PY{l+m+mi}{5} \PY{o}{*} \PY{n}{counter}\PY{p}{)}\PY{p}{,} 
                      \PY{n}{ax}\PY{o}{=}\PY{n}{axs}\PY{p}{,}
                      \PY{n}{label}\PY{o}{=}\PY{l+s+s2}{\PYZdq{}}\PY{l+s+s2}{hello}\PY{l+s+s2}{\PYZdq{}}\PY{p}{)}
         \PY{n}{plt}\PY{o}{.}\PY{n}{show}\PY{p}{(}\PY{p}{)}
\end{Verbatim}

    \begin{center}
    \adjustimage{max size={0.9\linewidth}{0.9\paperheight}}{output_14_0.png}
    \end{center}
    { \hspace*{\fill} \\}
    
    \begin{center}
    \adjustimage{max size={0.9\linewidth}{0.9\paperheight}}{output_14_1.png}
    \end{center}
    { \hspace*{\fill} \\}
    
    \begin{center}
    \adjustimage{max size={0.9\linewidth}{0.9\paperheight}}{output_14_2.png}
    \end{center}
    { \hspace*{\fill} \\}
    
    \begin{center}
    \adjustimage{max size={0.9\linewidth}{0.9\paperheight}}{output_14_3.png}
    \end{center}
    { \hspace*{\fill} \\}
    
    \begin{center}
    \adjustimage{max size={0.9\linewidth}{0.9\paperheight}}{output_14_4.png}
    \end{center}
    { \hspace*{\fill} \\}
    
    \subsection{Question 6.a \&6.b}\label{question-6.a-6.b}

Next, each year, temp is broken down into subdivisions in the range:
\[[minimum-0.1=t_0,maximum+0.1=t_10]\] into ten equal intervals. This is
then used to generate 10 intervals, as: \[[[t_0,t_1],...,[t_9,t_10]]\]

\begin{enumerate}
\def\labelenumi{\arabic{enumi}.}
\tightlist
\item
  binCorours are predefined RGB values, given in the question
\item
  The linspace function below is used to generate a set of numbers with
  equal distributions between the min and max data for each year.
\item
  Digitisation is used to put the data into the respective buckets. This
  can then be used later on when the graphs are drawn
\item
  generate a matrix to store the image colour values
\item
  iterate over each day of the year*2. double as we need to have two
  rows for each day(high and low)
\item
  set the current iteration-1 row of pixels to the low digitised index
  value from colour matrix provided.
\item
  same as above, but for the current iterator for the high values
  provided
\item
  convert matrix to image
\item
  draw image to screen
\end{enumerate}

The output below this code shows first the bins for each year, for
question 6.a. Then, the blankets are drawn for question 6.b

    \begin{Verbatim}[commandchars=\\\{\}]
{\color{incolor}In [{\color{incolor}8}]:} \PY{n}{binColours} \PY{o}{=} \PY{p}{[}\PY{p}{[}\PY{l+m+mf}{0.139681}\PY{p}{,} \PY{l+m+mf}{0.311666}\PY{p}{,} \PY{l+m+mf}{0.550652}\PY{p}{]}\PY{p}{,} \PY{p}{[}\PY{l+m+mf}{0.276518}\PY{p}{,} \PY{l+m+mf}{0.539432}\PY{p}{,} \PY{l+m+mf}{0.720771}\PY{p}{]}\PY{p}{,}
                      \PY{p}{[}\PY{l+m+mf}{0.475102}\PY{p}{,} \PY{l+m+mf}{0.695344}\PY{p}{,} \PY{l+m+mf}{0.802081}\PY{p}{]}\PY{p}{,} \PY{p}{[}\PY{l+m+mf}{0.670448}\PY{p}{,} \PY{l+m+mf}{0.803486}\PY{p}{,} \PY{l+m+mf}{0.824645}\PY{p}{]}\PY{p}{,}
                      \PY{p}{[}\PY{l+m+mf}{0.809791}\PY{p}{,} \PY{l+m+mf}{0.848259}\PY{p}{,} \PY{l+m+mf}{0.777550}\PY{p}{]}\PY{p}{,} \PY{p}{[}\PY{l+m+mf}{0.861927}\PY{p}{,} \PY{l+m+mf}{0.803423}\PY{p}{,} \PY{l+m+mf}{0.673050}\PY{p}{]}\PY{p}{,}
                      \PY{p}{[}\PY{l+m+mf}{0.830690}\PY{p}{,} \PY{l+m+mf}{0.667645}\PY{p}{,} \PY{l+m+mf}{0.546349}\PY{p}{]}\PY{p}{,} \PY{p}{[}\PY{l+m+mf}{0.742023}\PY{p}{,} \PY{l+m+mf}{0.475176}\PY{p}{,} \PY{l+m+mf}{0.424114}\PY{p}{]}\PY{p}{,}
                      \PY{p}{[}\PY{l+m+mf}{0.613033}\PY{p}{,} \PY{l+m+mf}{0.281826}\PY{p}{,} \PY{l+m+mf}{0.306352}\PY{p}{]}\PY{p}{,} \PY{p}{[}\PY{l+m+mf}{0.450385}\PY{p}{,} \PY{l+m+mf}{0.157961}\PY{p}{,} \PY{l+m+mf}{0.217975}\PY{p}{]}\PY{p}{]}
        \PY{n}{counter} \PY{o}{=} \PY{l+m+mi}{0}\PY{p}{;}
        \PY{k}{for} \PY{n}{year} \PY{o+ow}{in} \PY{n}{weatherData}\PY{p}{:}
            \PY{n}{binsLow} \PY{o}{=} \PY{n}{np}\PY{o}{.}\PY{n}{linspace}\PY{p}{(}\PY{n}{year}\PY{p}{[}\PY{p}{:}\PY{p}{,} \PY{l+m+mi}{3}\PY{p}{]}\PY{o}{.}\PY{n}{min}\PY{p}{(}\PY{p}{)}\PY{o}{\PYZhy{}}\PY{l+m+mf}{0.1}\PY{p}{,} \PY{n}{year}\PY{p}{[}\PY{p}{:}\PY{p}{,} \PY{l+m+mi}{3}\PY{p}{]}\PY{o}{.}\PY{n}{max}\PY{p}{(}\PY{p}{)}\PY{o}{+}\PY{l+m+mf}{0.1}\PY{p}{,} \PY{n}{num}\PY{o}{=}\PY{l+m+mi}{10}\PY{p}{)} 
            \PY{n}{binsHigh} \PY{o}{=} \PY{n}{np}\PY{o}{.}\PY{n}{linspace}\PY{p}{(}\PY{n}{year}\PY{p}{[}\PY{p}{:}\PY{p}{,} \PY{l+m+mi}{1}\PY{p}{]}\PY{o}{.}\PY{n}{min}\PY{p}{(}\PY{p}{)}\PY{o}{\PYZhy{}}\PY{l+m+mf}{0.1}\PY{p}{,} \PY{n}{year}\PY{p}{[}\PY{p}{:}\PY{p}{,} \PY{l+m+mi}{1}\PY{p}{]}\PY{o}{.}\PY{n}{max}\PY{p}{(}\PY{p}{)}\PY{o}{+}\PY{l+m+mf}{0.1}\PY{p}{,} \PY{n}{num}\PY{o}{=}\PY{l+m+mi}{10}\PY{p}{)}
                
            
            
            
            \PY{n}{digitizedLow}\PY{o}{=}\PY{n}{np}\PY{o}{.}\PY{n}{digitize}\PY{p}{(}\PY{n}{year}\PY{p}{[}\PY{p}{:}\PY{p}{,} \PY{l+m+mi}{3}\PY{p}{]}\PY{p}{,}\PY{n}{binsLow}\PY{p}{)} \PY{c+c1}{\PYZsh{}put the data into the bins}
            \PY{n}{digitizedHigh}\PY{o}{=}\PY{n}{np}\PY{o}{.}\PY{n}{digitize}\PY{p}{(}\PY{n}{year}\PY{p}{[}\PY{p}{:}\PY{p}{,} \PY{l+m+mi}{3}\PY{p}{]}\PY{p}{,}\PY{n}{binsHigh}\PY{p}{)}
            
            \PY{n}{rows}\PY{o}{=} \PY{l+m+mi}{2}\PY{o}{*}\PY{n+nb}{len}\PY{p}{(}\PY{n}{year}\PY{p}{[}\PY{p}{:}\PY{p}{,}\PY{l+m+mi}{3}\PY{p}{]}\PY{p}{)}
            \PY{n}{image} \PY{o}{=} \PY{n}{np}\PY{o}{.}\PY{n}{zeros}\PY{p}{(}\PY{p}{(}\PY{n}{rows}\PY{p}{,} \PY{l+m+mi}{451}\PY{p}{,} \PY{l+m+mi}{3}\PY{p}{)}\PY{p}{)}  \PY{c+c1}{\PYZsh{}make a matrix to store the image}
        
            \PY{k}{for} \PY{n}{x} \PY{o+ow}{in} \PY{n+nb}{range}\PY{p}{(}\PY{l+m+mi}{0}\PY{p}{,}\PY{n}{rows}\PY{p}{,}\PY{l+m+mi}{2}\PY{p}{)}\PY{p}{:} \PY{c+c1}{\PYZsh{}itterate over each year\PYZsq{}s values from the above values and set pixels colours}
                \PY{n}{image}\PY{p}{[}\PY{n}{x}\PY{o}{\PYZhy{}}\PY{l+m+mi}{1}\PY{p}{,}\PY{l+m+mi}{0}\PY{p}{:}\PY{l+m+mi}{451}\PY{p}{]}\PY{o}{=}\PY{n}{binColours}\PY{p}{[}\PY{n+nb}{int}\PY{p}{(}\PY{n}{digitizedLow}\PY{p}{[}\PY{n+nb}{int}\PY{p}{(}\PY{n}{x}\PY{o}{/}\PY{l+m+mi}{2}\PY{p}{)}\PY{p}{]}\PY{p}{)}\PY{p}{]}
                \PY{n}{image}\PY{p}{[}\PY{n}{x}\PY{p}{,}\PY{l+m+mi}{0}\PY{p}{:}\PY{l+m+mi}{451}\PY{p}{]}\PY{o}{=}\PY{n}{binColours}\PY{p}{[}\PY{n+nb}{int}\PY{p}{(}\PY{n}{digitizedHigh}\PY{p}{[}\PY{n+nb}{int}\PY{p}{(}\PY{n}{x}\PY{o}{/}\PY{l+m+mi}{2}\PY{p}{)}\PY{p}{]}\PY{p}{)}\PY{p}{]}
            \PY{n}{printText}\PY{p}{(}\PY{l+s+s2}{\PYZdq{}}\PY{l+s+s2}{\PYZlt{}h2\PYZgt{}Tempreture blanket for year: }\PY{l+s+si}{\PYZob{}\PYZcb{}}\PY{l+s+s2}{\PYZlt{}/h2\PYZgt{}}\PY{l+s+s2}{\PYZdq{}}\PY{o}{.}\PY{n}{format}\PY{p}{(}\PY{l+m+mi}{1995}\PY{o}{+}\PY{l+m+mi}{5}\PY{o}{*}\PY{n}{counter}\PY{p}{)}\PY{p}{)}
            
            \PY{n}{displayHTML}\PY{p}{(}\PY{n}{pd}\PY{o}{.}\PY{n}{DataFrame}\PY{p}{(}\PY{n}{np}\PY{o}{.}\PY{n}{column\PYZus{}stack}\PY{p}{(}\PY{p}{(}\PY{n}{binsLow}\PY{o}{.}\PY{n}{reshape}\PY{p}{(}\PY{l+m+mi}{10}\PY{p}{,}\PY{l+m+mi}{1}\PY{p}{)}\PY{p}{,}
                                                      \PY{p}{(}\PY{n}{binsHigh}\PY{o}{.}\PY{n}{reshape}\PY{p}{(}\PY{l+m+mi}{10}\PY{p}{,}\PY{l+m+mi}{1}\PY{p}{)}\PY{p}{)}\PY{p}{)}\PY{p}{)}\PY{p}{,} 
                                     \PY{n}{columns}\PY{o}{=}\PY{p}{[}\PY{l+s+s2}{\PYZdq{}}\PY{l+s+s2}{Low bin}\PY{l+s+s2}{\PYZdq{}}\PY{p}{,}\PY{l+s+s2}{\PYZdq{}}\PY{l+s+s2}{High bin}\PY{l+s+s2}{\PYZdq{}}\PY{p}{]}\PY{p}{)}\PY{o}{.}\PY{n}{to\PYZus{}html}\PY{p}{(}\PY{p}{)}\PY{p}{)}
            
            \PY{n}{outputImage} \PY{o}{=} \PY{n}{smp}\PY{o}{.}\PY{n}{toimage}\PY{p}{(}\PY{n}{image}\PY{p}{)}
            \PY{n}{outputImage}\PY{o}{.}\PY{n}{save}\PY{p}{(}\PY{l+s+s1}{\PYZsq{}}\PY{l+s+s1}{WeatherBlancketsOutput/}\PY{l+s+si}{\PYZob{}\PYZcb{}}\PY{l+s+s1}{.png}\PY{l+s+s1}{\PYZsq{}}\PY{o}{.}\PY{n}{format}\PY{p}{(}\PY{l+m+mi}{1995}\PY{o}{+}\PY{l+m+mi}{5}\PY{o}{*}\PY{n}{counter}\PY{p}{)}\PY{p}{)}
            \PY{n}{drawImg}\PY{p}{(}\PY{n}{outputImage}\PY{p}{)} \PY{c+c1}{\PYZsh{}Draw image to screen, using custom draw function to put output in window}
            \PY{n}{counter}\PY{o}{=}\PY{n}{counter}\PY{o}{+}\PY{l+m+mi}{1}\PY{p}{;}
\end{Verbatim}

    
    \begin{verbatim}
<IPython.core.display.HTML object>
    \end{verbatim}

    
    
    \begin{verbatim}
<IPython.core.display.HTML object>
    \end{verbatim}

    
    
    \begin{verbatim}
<IPython.core.display.HTML object>
    \end{verbatim}

    
    
    \begin{verbatim}
<IPython.core.display.HTML object>
    \end{verbatim}

    
    
    \begin{verbatim}
<IPython.core.display.HTML object>
    \end{verbatim}

    
    
    \begin{verbatim}
<IPython.core.display.HTML object>
    \end{verbatim}

    
    
    \begin{verbatim}
<IPython.core.display.HTML object>
    \end{verbatim}

    
    
    \begin{verbatim}
<IPython.core.display.HTML object>
    \end{verbatim}

    
    
    \begin{verbatim}
<IPython.core.display.HTML object>
    \end{verbatim}

    
    
    \begin{verbatim}
<IPython.core.display.HTML object>
    \end{verbatim}

    
    
    \begin{verbatim}
<IPython.core.display.HTML object>
    \end{verbatim}

    
    
    \begin{verbatim}
<IPython.core.display.HTML object>
    \end{verbatim}

    
    
    \begin{verbatim}
<IPython.core.display.HTML object>
    \end{verbatim}

    
    
    \begin{verbatim}
<IPython.core.display.HTML object>
    \end{verbatim}

    
    
    \begin{verbatim}
<IPython.core.display.HTML object>
    \end{verbatim}

    
    \subsection{Question 7}\label{question-7}

Next, a surface plot of a stochastic process is generated. For this,
avarages are generated over all 5 sets of data for each year, as well as
a standard diviation for each day. Next, a linear space is generated
representing the posible tempreture ranges. finally, the PDF is found
then plotted in 3D.

    \begin{Verbatim}[commandchars=\\\{\}]
{\color{incolor}In [{\color{incolor}9}]:} \PY{n}{plt}\PY{o}{.}\PY{n}{rcParams}\PY{p}{[}\PY{l+s+s1}{\PYZsq{}}\PY{l+s+s1}{figure.figsize}\PY{l+s+s1}{\PYZsq{}}\PY{p}{]} \PY{o}{=} \PY{p}{(}\PY{l+m+mi}{24}\PY{p}{,} \PY{l+m+mi}{6}\PY{p}{)} \PY{c+c1}{\PYZsh{}Make the figure for this question bigger}
        
        \PY{c+c1}{\PYZsh{}define the variables to store the mean, std deviation, range for each day(temp), pdfs and a vector for number of days}
        \PY{n}{meanDay}\PY{o}{=}\PY{n}{np}\PY{o}{.}\PY{n}{zeros}\PY{p}{(}\PY{p}{(}\PY{l+m+mi}{365}\PY{p}{,}\PY{l+m+mi}{1}\PY{p}{)}\PY{p}{)}
        \PY{n}{stdDay}\PY{o}{=}\PY{n}{np}\PY{o}{.}\PY{n}{zeros}\PY{p}{(}\PY{p}{(}\PY{l+m+mi}{365}\PY{p}{,}\PY{l+m+mi}{1}\PY{p}{)}\PY{p}{)}
        \PY{n}{dayArray}\PY{o}{=}\PY{n}{np}\PY{o}{.}\PY{n}{zeros}\PY{p}{(}\PY{p}{(}\PY{l+m+mi}{5}\PY{p}{,}\PY{l+m+mi}{1}\PY{p}{)}\PY{p}{)}
        \PY{n}{pdf}\PY{o}{=}\PY{n}{np}\PY{o}{.}\PY{n}{zeros}\PY{p}{(}\PY{p}{(}\PY{l+m+mi}{365}\PY{p}{,}\PY{l+m+mi}{365}\PY{p}{)}\PY{p}{)}
        \PY{n}{dayRange}\PY{o}{=}\PY{n+nb}{range}\PY{p}{(}\PY{l+m+mi}{0}\PY{p}{,}\PY{l+m+mi}{365}\PY{p}{,}\PY{l+m+mi}{1}\PY{p}{)}
        
        \PY{k}{for} \PY{n}{day} \PY{o+ow}{in} \PY{n}{dayRange}\PY{p}{:}
            \PY{k}{for} \PY{n}{inx}\PY{p}{,} \PY{n}{year} \PY{o+ow}{in} \PY{n+nb}{enumerate}\PY{p}{(}\PY{n}{weatherData}\PY{p}{)}\PY{p}{:}
                \PY{n}{dayArray}\PY{p}{[}\PY{n}{inx}\PY{p}{,}\PY{l+m+mi}{0}\PY{p}{]}\PY{o}{=}\PY{n}{year}\PY{p}{[}\PY{p}{:}\PY{p}{,} \PY{l+m+mi}{2}\PY{p}{]}\PY{p}{[}\PY{n}{day}\PY{p}{]}
            \PY{n}{meanDay}\PY{p}{[}\PY{n}{day}\PY{p}{]}\PY{o}{=}\PY{n}{dayArray}\PY{o}{.}\PY{n}{mean}\PY{p}{(}\PY{p}{)} \PY{c+c1}{\PYZsh{}calculate the mean}
            \PY{n}{stdDay}\PY{p}{[}\PY{n}{day}\PY{p}{]}\PY{o}{=}\PY{n}{dayArray}\PY{o}{.}\PY{n}{std}\PY{p}{(}\PY{p}{)} \PY{c+c1}{\PYZsh{} calculate the standard deviation}
        
        \PY{c+c1}{\PYZsh{}generate a linear space of all days in the region, for the min to max temp}
        \PY{n}{tempretureSpace}\PY{o}{=}\PY{n}{np}\PY{o}{.}\PY{n}{linspace}\PY{p}{(}\PY{n}{meanDay}\PY{o}{.}\PY{n}{min}\PY{p}{(}\PY{p}{)}\PY{p}{,}
                                    \PY{n}{meanDay}\PY{o}{.}\PY{n}{max}\PY{p}{(}\PY{p}{)}\PY{p}{,}
                                    \PY{n+nb}{len}\PY{p}{(}\PY{n}{weatherData}\PY{p}{[}\PY{l+m+mi}{0}\PY{p}{]}\PY{p}{[}\PY{p}{:}\PY{p}{,}\PY{l+m+mi}{2}\PY{p}{]}\PY{p}{)}\PY{p}{)}
        
        \PY{c+c1}{\PYZsh{}itterate over the days again, now generating the pdf}
        \PY{k}{for} \PY{n}{day} \PY{o+ow}{in} \PY{n}{dayRange}\PY{p}{:}
            \PY{n}{pdf}\PY{p}{[}\PY{n}{day}\PY{p}{,}\PY{p}{:}\PY{p}{]}\PY{o}{=}\PY{n}{norm}\PY{o}{.}\PY{n}{pdf}\PY{p}{(}\PY{n}{tempretureSpace}\PY{p}{,}
                                \PY{n}{meanDay}\PY{p}{[}\PY{n}{day}\PY{p}{]}\PY{p}{,}
                                \PY{n}{stdDay}\PY{p}{[}\PY{n}{day}\PY{p}{]}\PY{p}{)}
        
        \PY{c+c1}{\PYZsh{}convert the values to a meshgrid (return coordinate matrices from coordinate vectors)}
        \PY{n}{tempretureSpace}\PY{p}{,} \PY{n}{dayRange} \PY{o}{=} \PY{n}{np}\PY{o}{.}\PY{n}{meshgrid}\PY{p}{(}\PY{n}{tempretureSpace}\PY{p}{,} \PY{n}{dayRange}\PY{p}{)}
        
        \PY{c+c1}{\PYZsh{}finally, plot it as a 3d surf}
        \PY{n}{fig} \PY{o}{=} \PY{n}{plt}\PY{o}{.}\PY{n}{figure}\PY{p}{(}\PY{p}{)}
        \PY{n}{ax} \PY{o}{=} \PY{n}{fig}\PY{o}{.}\PY{n}{gca}\PY{p}{(}\PY{n}{projection}\PY{o}{=}\PY{l+s+s1}{\PYZsq{}}\PY{l+s+s1}{3d}\PY{l+s+s1}{\PYZsq{}}\PY{p}{)}
        \PY{n}{surf} \PY{o}{=} \PY{n}{ax}\PY{o}{.}\PY{n}{plot\PYZus{}surface}\PY{p}{(}\PY{n}{tempretureSpace}\PY{p}{,}
                               \PY{n}{dayRange}\PY{p}{,} \PY{n}{pdf}\PY{p}{,}
                               \PY{n}{cmap}\PY{o}{=}\PY{n}{cm}\PY{o}{.}\PY{n}{coolwarm}\PY{p}{,}
                               \PY{n}{linewidth}\PY{o}{=}\PY{l+m+mi}{0}\PY{p}{,}
                               \PY{n}{antialiased}\PY{o}{=}\PY{k+kc}{False}\PY{p}{)}
        
        \PY{n}{ax}\PY{o}{.}\PY{n}{set\PYZus{}xlabel}\PY{p}{(}\PY{l+s+s1}{\PYZsq{}}\PY{l+s+s1}{Temperature (C)}\PY{l+s+s1}{\PYZsq{}}\PY{p}{)}
        \PY{n}{ax}\PY{o}{.}\PY{n}{set\PYZus{}ylabel}\PY{p}{(}\PY{l+s+s1}{\PYZsq{}}\PY{l+s+s1}{Days}\PY{l+s+s1}{\PYZsq{}}\PY{p}{)}
        \PY{n}{ax}\PY{o}{.}\PY{n}{set\PYZus{}zlabel}\PY{p}{(}\PY{l+s+s1}{\PYZsq{}}\PY{l+s+s1}{f(x)}\PY{l+s+s1}{\PYZsq{}}\PY{p}{)}
        
        \PY{c+c1}{\PYZsh{}add a colour bar on the side to see magnitudes}
        \PY{n}{fig}\PY{o}{.}\PY{n}{colorbar}\PY{p}{(}\PY{n}{surf}\PY{p}{,} \PY{n}{shrink}\PY{o}{=}\PY{l+m+mf}{0.5}\PY{p}{,} \PY{n}{aspect}\PY{o}{=}\PY{l+m+mi}{5}\PY{p}{)}
        \PY{n}{plt}\PY{o}{.}\PY{n}{title}\PY{p}{(}\PY{l+s+s1}{\PYZsq{}}\PY{l+s+s1}{Surface Plot of Stochastic Process}\PY{l+s+s1}{\PYZsq{}}\PY{p}{)}
        \PY{n}{plt}\PY{o}{.}\PY{n}{show}\PY{p}{(}\PY{p}{)}
\end{Verbatim}

    \begin{center}
    \adjustimage{max size={0.9\linewidth}{0.9\paperheight}}{output_18_0.png}
    \end{center}
    { \hspace*{\fill} \\}
    

    % Add a bibliography block to the postdoc
    
    
    
    \end{document}
